\documentclass{beamer}
\usetheme{default}
\usecolortheme{whale}

\usepackage[frenchb]{babel}
\usepackage[utf8]{inputenc}
\usepackage[T1]{fontenc}

\usepackage{color}


\usepackage{pgfpages}
%\setbeameroption{show notes on second screen}
\setbeamertemplate{note page}[default]

\title{Projet Chaos}
\subtitle{Billard Carré avec Barre Centrale}
\author[J. Chi, N. Dwek]{Jun Nuo Chi, Nathan Dwek}
\institute{Ecole Polytechnique de Bruxelles}
\date{8 janvier 2014}
\logo{\includegraphics[height=0.8cm]{./EPB.jpg}}

\begin{document}

  \frame{\titlepage}
  
  \section{Introduction}
  \begin{frame}{Introduction}
  \framesubtitle{Théorie du Chaos - But du Projet}
  \begin{itemize}
    \item Système déterministe mais non prédictible à long terme
    \begin{itemize}
      \item Possède des équations d'évolution déterministes
      \pause \item Sensible aux conditions initiales
      \item Non linéaire (superposition non applicable)
    \end{itemize}
    \pause \item Applications dans de nombreux domaines: météorologie, finance, mécanique \ldots
    \pause \item Etude du mouvement d'une balle dans un billard carré muni d'une barre centrale respirante en fonction des paramètres du système:
    \begin{itemize}
      \pause \item Orientation du billard: vertical ou horizontal 
      \item Paramètres de respiration de la barre: \(l={\color{red}l_0}(1+sin({\color{red}\omega} t))\)
      \item Conditions initiales de la balle: position et vitesse initiales
    \end{itemize}
  \end{itemize}
  \end{frame}

  \section{Modélisation}

  \begin{frame}{Modélisation}
  \framesubtitle{Modélisation du Mouvement et des Rebonds - Résolution Numérique}
  \begin{itemize}
    \item Mouvement composé d'une suite de déplacement continus:\\
    \pause \begin{center}
      \(\begin{cases}
        {\textstyle \ddot{x}=0}\\
        {\textstyle \ddot{y}=-g}\\
      \end{cases}\)
    \end{center}
    \pause \item Déplacement interrompu par un rebond qui définit les conditions initiales pour le déplacement suivant
    \begin{itemize}
      \pause \item Rebond sur une paroi extérieure du billard:
      \begin{itemize}
        \item \(x=\pm \frac{L}{2} \text{ ou } y=\pm \frac{L}{2}\)
        \item Simple inversion de la vitesse selon une des coordonnées 
      \end{itemize}
      \pause \item Rebond sur la barre centrale:
      \begin{itemize}
        \item \(\lvert x \rvert \leq l_0(1+sin(\omega t)) \text{ et } y=0\)
        \item Transfert de quantité de mouvement avec \({\scriptstyle m_{barre}>>m_{balle}}\):
        \[\begin{cases}
          \dot{x}^+=C\dot{x}^- +  (\text{sgn}(x))(1+C)cos(\omega t)\omega\\
          \dot{y}^+=-C \dot{y}^-
        \end{cases}\]
      \end{itemize}
    \end{itemize}
  \end{itemize}
  \end{frame}

  \begin{frame}{Modélisation}
  \framesubtitle{Considérations Théoriques}
  \begin{itemize}
    \item Pas de transfert de quantité de mouvement en x \(\rightleftharpoons\) y ou système \(\rightleftharpoons\) y
     \begin{itemize}
       \item Si \(g=0\): Conservation de \(\lvert \dot{y} \rvert\)
       \item Si \(g\neq0\): Conservation de \(y_{max}=\frac{\dot{y}^2}{2}+gy\)
       \begin{itemize}
         \item Zone \(y>y_{max}\) inaccessible
         \pause \item Cas dégénéré \(y_{max} \leq 0\): pas d'interaction avec la barre
         \item Cas dégénéré \(y_{max} \gg \frac{L}{2}\): influence de la gravité négligeable
       \end{itemize}
     \end{itemize}
    \pause \item Mouvements en x et en y quasi indépendants
    \item Identification des sources probables de chaos
    \begin{itemize}
      \pause \item Chaos en x \(\Rightarrow\) chaos en y
      \item Barre au repos \(\Rightarrow\) mouvement en x régulier
      \pause \item Chaos en x \(\overset{?}{\Leftrightarrow}\) chaos en y \(\rightarrow\) A vérifier!
    \end{itemize}
  \end{itemize}
  \end{frame}
  
  \section{Barre Centrale au Repos}
  
  \begin{frame}{Barre Centrale au Repos}
  \framesubtitle{Observations}
  \begin{itemize}
    \item Billard horizontal:
    \begin{itemize}
      \item Mouvement régulier en x et en y comme attendu
      \item Deux états échantillonables en y qui s'enchaînent de manière régulière
    \end{itemize}
    \pause \item Billard vertical:
    \begin{itemize}
      \item Mouvement toujours régulier en x
      \pause \item Mouvement en y:
    \end{itemize}
  \end{itemize}
  \includegraphics[width=0.79\textwidth]{bifYG981W0.png}
  \end{frame}
  
  \begin{frame}{Barre Centrale au Repos}
  \framesubtitle{Interprêtation dans le Cas Billard Vertical}
  \begin{itemize}
    \item Mouvement formé d'une suite de trois ``cycles'' dont deux de longueur indépendante en y
    \begin{itemize}
      \item Infinité d'états échantillonables
      \pause \item Période potentielle = combili naturelle des longueurs de ces trois cycles
      \begin{itemize}
        \item Vérifié par les simulations
        \item Période peut être très longue \(\Rightarrow\) indicateur de la transition vers le chaos
        \item Mais une telle période ne semble pas toujours exister
        \item Explication: "période irrationnelle"
      \end{itemize}
    \end{itemize}
    \pause \item \(l=L\): mouvement périodique dans un demi billard \(\Rightarrow\) période différente 
    \item Cas dégénéré \(y_{max} \gg \frac{+L}{2}\) et \( y_{max}<0 \) également vérifiés par les simulations
  \end{itemize}
  \end{frame}

  \section{Barre Centrale Respirante}
  
  \begin{frame}{Barre Centrale Respirante}
  \framesubtitle{Billard Horizontal}
  \begin{itemize}
    \item Mouvement en x transitionne très rapidement vers le chaos
    \begin{itemize}
      \item Paramètre représentatif: \(\frac{l_0 \omega ^ 2}{\dot{x}_0}\)
      \pause \item Transition pour \(\frac{l_0 \omega ^ 2}{\dot{x}_0} \simeq 10^{-7} \)
    \end{itemize}
  \end{itemize}
  \includegraphics[width=0.9\textwidth]{bifXG981WTrans.png}
  \end{frame}
  
  \begin{frame}{Barre Centrale Respirante}
  \framesubtitle{Billard Horizontal}
  \begin{itemize}
    \item Mouvement en y:
    \begin{itemize}
      \item Théorie: Mouvement en x chaotique \( \Rightarrow \) Mouvement en y chaotique
      \item Observation: mouvement en x périodique \( \Leftrightarrow l_0 \omega ^ 2 \ll \dot{x}_0 \)\\ \( \Rightarrow \text{Se ramène au cas } g=0, \omega =0\) \\ \( \Rightarrow \text{Mouvement en y périodique} \)
      \pause \item Mais toujours 2 états échantillonnables (quel que soit le mouvement en x)\\\(\Rightarrow\) Pas apparent sur un diagramme de bifurcation ou section de poincaré
      \item 2 états peuvent s'enchaîner de manière chaotique\\\(\Rightarrow\) Vérification expérimentable possible par la coordonnée y en fonction du temps.
    \end{itemize}
  \end{itemize}
  \end{frame}
  
    \begin{frame}{Barre Centrale Respirante}
  \framesubtitle{Billard Vertical}
  \begin{itemize}
    \item Transition similaire du mouvement en x vers le chaos
    \item Mouvement en y
    \begin{itemize}
      \item Théorie: Mouvement en x chaotique \( \Rightarrow \) Mouvement en y chaotique
      \item Observation: mouvement en x périodique \( \Leftrightarrow l_0 \omega ^ 2 \ll \dot{x}_0 \)\\ \( \Rightarrow \text{Se ramène au cas } g=9.81, \omega =0\)\\
      \( \Rightarrow\) Conclusions de ce cas à réutiliser.
    \end{itemize}
  \end{itemize}
  \end{frame}
  
  \section{Conclusion}
  
  \begin{frame}{Conclusion}
  \begin{itemize}
    \item Barre toujours responsable du chaos
    \item Deux sortes de chaos relativement différentes en x et en y
    \begin{itemize}
      \item Chaos en y assez binaire et dépend fortement du comportement en x
      \item Conservation de \(y_{max}\) quoi qu'il arrive et notion de ``déphasage'' pour l'étude du mouvement en y
      \pause \item Chaos en x présente une transition.
      \item Pas de conservation de l'énergie cinétique pour l'étude du mouvement en x
    \end{itemize}
    \item Utile d'étudier les lois de conservation du système pour prédire et expliquer le comportement du système et les cas dégénérés
  \end{itemize}
  \begin{center}
  \pause {\rmfamily{\scshape Merci de votre attention}}
  \end{center}
  \end{frame}
  
\end{document}
