\documentclass{beamer}
\usepackage[utf8]{inputenc}
\usetheme{PaloAlto}
\usecolortheme{seahorse}
\usepackage{color}



\title{Projet Chaos}
\subtitle{Billard Carré avec Barre Centrale}
\author[J. Chi, N. Dwek]{Jun Nuo Chi, Nathan Dwek}
%\institute{Université Libre de Bruxelles}
\date{8 janvier 2014}
\begin{document}
  \frame{\titlepage}
  \section{Introduction}
  \begin{frame}{Introduction}
  \framesubtitle{Théorie du Chaos - But du Projet}
  \begin{itemize}
    \item Système déterministe mais non prédictible à long terme
    \begin{itemize}
      \item Possède des équations d'évolution déterministes
      \pause \item Sensible aux conditions initiales
      \item Non linéaire (superposition non applicable)
    \end{itemize}
    \pause \item Applications dans de nombreux domaines: météorologie, finance, mécanique \ldots
    \pause \item Etude du mouvement d'une balle dans un billard carré muni d'une barre centrale respirante en fonction des paramètres du système:
    \begin{itemize}
      \item Orientation du billard: vertical ou horizontal 
      \item Paramètres de respiration de la barre: \(l={\color{red}l_0}(1+sin({\color{red}\omega} t))\)
      \item Conditions initiales de la balle: position et vitesse initiales
    \end{itemize}
  \end{itemize}
  \end{frame}
  \section{Modélisation}
  \begin{frame}{Modélisation}
  \framesubtitle{Modélisation du Mouvement et des Rebonds - Résolution Numérique}
  \begin{itemize}
    \item Mouvement composé d'une suite de déplacement continus:\pause
    \begin{equation*}
      \begin{cases}
        \ddot{x}=0\\
        \ddot{y}=-g\\
      \end{cases}
    \end{equation*}
    \pause \item
  \end{itemize}

  \end{frame}
  \begin{frame}{Modélisation}
  \framesubtitle{Considérations Théoriques}  
  \end{frame}
  \section{Barre Centrale au Repos}
  \begin{frame}
  \end{frame}
  \section{Barre Centrale Respirante}
  \begin{frame}
  \end{frame}
  \section{Conclusion}
  \begin{frame}
  \end{frame}
\end{document}
