\documentclass{beamer}
\usepackage[utf8]{inputenc}
\usetheme{Goettingen}
\usecolortheme{seahorse}
\usepackage{color}



\title{Projet Chaos}
\subtitle{Billard Carré avec Barre Centrale}
\author[J. Chi, N. Dwek]{Jun Nuo Chi, Nathan Dwek}
\institute{Ecole Polytechnique de Bruxelles}
\date{8 janvier 2014}
\begin{document}
  \frame{\titlepage}
  \section{Introduction}
  \begin{frame}{Introduction}
  \framesubtitle{Théorie du Chaos - But du Projet}
  \begin{itemize}
    \item Système déterministe mais non prédictible à long terme
    \begin{itemize}
      \item Possède des équations d'évolution déterministes
      \pause \item Sensible aux conditions initiales
      \item Non linéaire (superposition non applicable)
    \end{itemize}
    \pause \item Applications dans de nombreux domaines: météorologie, finance, mécanique \ldots
    \pause \item Etude du mouvement d'une balle dans un billard carré muni d'une barre centrale respirante en fonction des paramètres du système:
    \begin{itemize}
      \pause \item Orientation du billard: vertical ou horizontal 
      \pause \item Paramètres de respiration de la barre: \(l={\color{red}l_0}(1+sin({\color{red}\omega} t))\)
      \pause \item Conditions initiales de la balle: position et vitesse initiales
    \end{itemize}
  \end{itemize}
  \end{frame}

  \section{Modélisation}

  \begin{frame}{Modélisation}
  \framesubtitle{Modélisation du Mouvement et des Rebonds - Résolution Numérique}
  \begin{itemize}
    \item Mouvement composé d'une suite de déplacement continus:\\
    \pause \begin{center}
      \(\begin{cases}
        {\textstyle \ddot{x}=0}\\
        {\textstyle \ddot{y}=-g}\\
      \end{cases}\)
    \end{center}
    \pause \item Déplacement interrompu par un rebond qui définit les conditions initiales pour le déplacement suivant
    \begin{itemize}
      \pause \item Rebond sur une paroi extérieure du billard:
      \begin{itemize}
        \item \(x=\pm \frac{L}{2} \text{ ou } y=\pm \frac{L}{2}\)
        \item Simple inversion de la vitesse selon une des coordonnées 
      \end{itemize}
      \pause \item Rebond sur la barre centrale:
      \begin{itemize}
        \item \(\lvert x \rvert \leq l_0(1+sin(\omega t)) \text{ et } y=0\)
        \item Transfert de quantité de mouvement avec \({\scriptstyle m_{barre}>>m_{balle}}\):
        \[\begin{cases}
          \dot{x}^+=C\dot{x}^- +  (\text{sgn}(x^*))(1+C)cos(\omega t^*)\omega\\
          \dot{y}^+=-C \dot{y}^-
        \end{cases}\]
      \end{itemize}
    \end{itemize}
  \end{itemize}
  \end{frame}


  \begin{frame}{Modélisation}
  \framesubtitle{Considérations Théoriques}
  \begin{itemize}
     \item Pas de transfert de quantité de mouvement en x \(\rightleftharpoons\) y ou système \(\rightleftharpoons\) y
     \begin{itemize}
       \item si \(g=0\): Conservation de \(\lvert \dot{y} \rvert\)
       \item si \(g\neq0\): Conservation de \(y_{max}=\frac{\dot{y}^2}{2}+gy\)
     \end{itemize}

  \end{itemize}

  \end{frame}
  
  \section{Barre Centrale au Repos}
  
  \begin{frame}
  \end{frame}
  
  \section{Barre Centrale Respirante}
  
  \begin{frame}
  \end{frame}
  
  \section{Conclusion}
  
  \begin{frame}
  \end{frame}
  
\end{document}
